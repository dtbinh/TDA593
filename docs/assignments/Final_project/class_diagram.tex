\section{Class Diagram}
The application have 4 different packages:
\begin{itemize}
    \item model - Data classes used for storage and communication.
    \item robot - Logic that controls a robot
    \item control\_station - Logic that orchestrates the robots.
    \item user\_interface - Interface that the user can use to communicate with the control\_station.
\end{itemize}
All packages except for model are loosely coupled to allow for different system configuration where they can be split up in different programs on different computers and use network communication to interact with each other. Model is used by all packages and for the communication between the packages and is therefore tied in to all the other packages. It should also be said that the robot have no connection to the user\_interface as all communication goes trough the control\_station.

The whole system can in its current configuration be initiated by calling the build method on the various Factory classes.

\subsection{Robot Package}
The robot package have a controller class which takes care of periodically collecting sensor information and performing actions. This is done with the Sensor and Actuator interfaces which SimulatorRobot implements by inheriting the AbstractRobot class from the simulator package. Another configuration may then use another implementation of Sensor and Actuator to control a real robot.
\\ \\
Routines are used to describe the logic, given a status for the robot they return an action which indicates what the robot should do. An action can then if given an actuator be executed. If the action is blocking then no other blocking action can be run. For example both move and emergency stop are blocking actions so only one of them can run but the colour action can run simultaneously.
\\ \\
ControlStationInterface is a interface for the control station which the control\_station package uses to communicate with the robot. This is done trough the getStatus and notify methods.

\subsection{Control Station Package}
The control\_station package orchestrates the robots and calculates the point based on the user input. Any user interface can get infomration about robots, points and environment as well as assign missions and actions to robots trough the OperatorInterface which then sends the missions to the conductor which handles the logic. Actions however bypass the conductor and go directly to the RobotInterface, allowing the user to send an emergency stop signal even if there is a fault in the conductor.
\\ \\
The conductor strategizes (order the points in) a mission with help of the Strategizer and tells the robot through simple instruction what to do to complete the mission. The conductor can at any time send new instructions to the robot if the mission changes or some kind of obstacle appears.
\\ \\
The RobotInterface keeps track of the robots and allow other parts of the control\_station to dispatch instructions to them. It also periodicly polls the robots for their status to make sure that the control\_station always have curretn status in its storage.
\\ \\
The PointRewarder keeps track of the current reward Procedure and periodically uses it to calculate points as well as calculating the next procedure through the current procedures update method.

\subsubsection{Storage Package}
The storage package contains and stores all the data regarding robots, environment, reward points, and missions. It also provides functionality to access and show this data to other parts of the system which may require or want to see this data.
\subsection{Model Package}
The model package contains all the different classes and data types that the system makes use of. Representations of instructions are instantiated as classes, and there are also classes representing everything inside an environment etc. We have our own representations of for example areas and walls, to make the system more extensible (to make it able to connect to many types of simulators and robots). All the other packages make use of this package.
\subsection{User Interface Package}
The user interface package contains the classes that regards logic and interaction with a user. It provides functions such as inputting new missions for robots to execute. This is basically the interface between the user and the robots.

\subsection{Worth noting}
The external simulator package is only displayed to a minimum in this class diagram as it's outside the scope. The diagram only shows the methods and classes in that package that our packages make use of. We decided to not include it further as its just a temporary package used for this prototype and has nothing to do with our system.
\\ \\
The think arrows to the model package indicates that many classes in each of the other packages use many classes in the model package. We thought it best to display the relations like this as the model is used as our storage and communication language within the application.
\\ \\
Most private variables are hidden in the class diagram. This is because they are obvious (have getters and setters) or doesn't matter to the bigger picture that the class diagram tires to show and only are a part of the internal logic of that class. 