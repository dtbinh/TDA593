\section{Sequence diagrams}
\subsection{Completing a mission}
  This diagram models the sequence of actions that occur between assigning a mission and having it completed by a robot. The selected abstraction level for timelines are package-bases, as our system is designed in a way that each package could be a detached program on a separate machine.
 \begin{itemize}
    \item Starts when the operator inputs the mission to the interface.
    \item The control station (Conductor) calculates a strategy for completing the given mission i.e sort the collection of mission points according to the given strategy.
    \item The control station sends one movement instruction at a time to the robot in order to have full control of the robot. It goes on to check if the robot has reached the next location, if that is the case the control station will send a the next location in order to progress in the mission. This procedure will be done in a loop until all the instructions are completed.
    \item Also in parallel, as the operator may want information about the robot, the control station will keep an updated storage which the user interface can query.
\end{itemize}
\subsection{Change of strategy}
In our system, changing the strategy of a mission is considered equivalent to assigning a new mission with a different strategy. This results in the sequence diagram being very similar to the previous one.
\begin{itemize}
    \item Starts when the operator inputs the mission to the interface.
    \item The control station (Conductor) calculates a strategy for completing the given mission i.e sort the collection of mission points according to the given strategy.
    \item The control station sends one movement instruction at a time to the robot in order to have full control of the robot. It goes on to check if the robot has reached the next location, if that is the case the control station will send a the next location in order to progress in the mission. This procedure will be done in a loop until all the instructions are completed.
    \item Also in parallel, as the operator may want information about the robot, the control station will keep an updated storage which the user interface can query.
\end{itemize}
\subsection{Calculate reward}
This final sequence diagram illustrates the process held within the \textit{Point Rewarder} in the control station.
\begin{itemize}
    \item The point rewarder process is initiated when the control station is started.
    \item In order to calculate the reward points, the point rewarder asks the storage package to get the locations of all robots and the areas in the environment.
    \item It asks the current procedure how many points it will reward given the locations and areas.
    \item The global stored points are increased in the storage module with the increment that was rewarded.
    \item Lastly, the procedure calculates whether the current one will stay or if it should change to another procedure.
    \item In this diagram the abstraction level is lower than the previous ones. The reason for this is that the sequence for this scenario is only in a single package. In order to show what is really going the abstraction level needed to be lower for this case.
\end{itemize}